\documentclass[10pt,prd,aps,nofootinbib,superscriptaddress]{revtex4}
\usepackage{epsf,graphicx,xcolor,amsmath}

\usepackage{lipsum}
%\newcommand{\textit}{\it}
\newcommand{\beq}{\begin{equation}}
\newcommand{\eeq}{\end{equation}}
\newcommand{\bea}{\begin{eqnarray}}
\newcommand{\eea}{\end{eqnarray}}
\newcommand{\nn}{\nonumber}



\begin{document}

\title{$\Upsilon$ photoproduction on the proton at an Electron-Ion Collider}
\author{Oleksii Gryniuk}
\affiliation{Institut f\"ur Kernphysik \& PRISMA$^+$  Cluster of Excellence, Johannes Gutenberg Universit\"at,  D-55099 Mainz, Germany}
\author{Sylvester Joosten}
\affiliation{Argonne National Laboratory, Lemont, IL 60439, USA}
\author{Zein-Eddine Meziani}
\affiliation{Argonne National Laboratory, Lemont, IL 60439, USA}
\author{Marc Vanderhaeghen}
\affiliation{Institut f\"ur Kernphysik \& PRISMA$^+$  Cluster of Excellence, Johannes Gutenberg Universit\"at,  D-55099 Mainz, Germany}
\noaffiliation
\date{\today}

\begin{abstract}
abstract...
\end{abstract}

\maketitle

\tableofcontents


\section{Introduction}

QUARKONIUM Interactions with hadrons

The interaction between heavy quarkonia, such as $J/\psi$ and $\Upsilon$, and light hadrons or nuclei provides a unique tool to test the gluonic van der Waals interaction in Quantum Chromo Dynamics (QCD). Being a small sized system, the heavy quarkonia can be treated as a color dipole, and the effective interaction between the quarkonium and the nucleon or nucleus may be estimated from the knowledge of its chromo-electric polarizability, see 
Refs.~\cite{Kharzeev:1995ij,Voloshin:2007dx,Hosaka:2016ypm} for reviews and references therein. 

If this effective interaction is strong enough, 
a bound state between the
$Q \bar Q$ state and a nucleus~\cite{Brodsky:1989jd, Wasson:1991fb, Luke:1992tm} may be formed. 
Using a perturbative calculation for the chromo-electric polarizability of a heavy quarkonium~\cite{Peskin:1979va} 
and a two-gluon exchange interaction, 
Ref.~\cite{Luke:1992tm} estimated a binding energy for a $J/\psi$ in nuclear matter $B_{J/\psi} \sim 10$~MeV.  

Within lattice QCD, the s-wave effective potentials for the  
$J/\Psi$-N system ($J^P = 1/2^-$ and $J^P = 3/2^-$) were recently studied~\cite{Sugiura:2017vks}. For the nucleon, the potential was found attractive, but not strong enough to allow for bound states. 

In a recent work~\cite{Polyakov:2018aey}, the chromoelectric polarizability of $J/\Psi$ was extracted from the lattice QCD data on the nucleon-$J/\Psi$ potential~\cite{Sugiura:2017vks}  in the heavy-quark limit. The value of $\alpha(\psi(1S)) = 1.6 \pm 0.8$~GeV$^{-3}$ was obtained.





HADROCHARMONIA:  TETRAQUARKS and PENTAQUARKS

Tetraquarks in charmonium and bottomonium

Pentaquarks: LHCb~\cite{Aaij:2015tga}

The pentaquark state $P_c(4450)$, discoverd by LHCb, was interpreted in Ref.~\cite{Eides:2015dtr} as a bound state of $\psi(2S)$ and the nucleon. The binding is due to chromoelectric interaction between a small quarkonium state and the nucleon.
Using a chiral-quark soliton model description of the nucleon mean field, it was found that the nucleon-$\psi(2S)$ bound state has a naturally narrow width in the range of tens of MeV.



In a previous work~\cite{Gryniuk:2016mpk}, a phenomenological analysis of the forward $J/\psi$-p scattering amplitude was performed within a dispersive framework, by relating the imaginary part of the forward $J/\psi$-p scattering amplitude 
to $\gamma p \to J/\psi p$ and $\gamma p \to c \bar c X$ cross section data, and by calculating the real part through a once-subtracted dispersion relation.  
The resulting dispersive framework extracted as value for the 
spin-averaged s-wave $\psi$-p scattering length $a_{J/\psi p} = 0.046 \pm 0.005$~fm, which can be translated into a 
$J/\psi$ binding energy in nuclear matter 
of $B_{J/\psi} = 2.7 \pm 0.3$~MeV. 

$J/\Psi$ Threshold photoproduction experiments: GLueX@JLab, HallC expt.


$\Upsilon$ production

PROTON MASS...gluonic contribution, trace of energy-momentum tensor...

EIC

HERA Data: \cite{Behnke:2015qja}

\newpage

\section{$\Upsilon$ proton forward scattering amplitude}

We consider the spin-averaged $\Upsilon p \to \Upsilon p$ forward elastic scattering process, which is described by an invariant amplitude $T_{\Upsilon p}$, depending on the crossing variable $\nu$. The latter is defined in terms of the Mandelstam invariant $s$ as:
\bea
\nu = \frac{1}{2} (s - M^2 - M_\Upsilon^2),
\eea
where $M (M_\Upsilon)$ stand for the masses of the proton $(\Upsilon)$ respectively.  
The forward differential cross section for the $\Upsilon \, p \to \Upsilon \, p$ scattering process can then be expressed as:
\bea
\frac{d \sigma}{dt} \biggr|_{t = 0} (\Upsilon p \to \Upsilon p) = \frac{1}{64 \, \pi \, s \, q_{\Upsilon p}^2} \, \big| T_{\Upsilon p}(\nu) \big|^2,
\eea
where in the forward direction the momentum transfer $t = 0$, and where $q_{\Upsilon p}$ denotes the magnitude of the $\Upsilon$ three-momentum in the c.m. frame, given by:
 \bea 
 q_{\Upsilon p}^2  = \frac{1}{4 s} \left[ s - (M_\Upsilon + M)^2 \right] \left[ s - (M_\Upsilon - M)^2 \right].
 \eea 
 
The imaginary part of the amplitude $T_{\Upsilon p}$ can be obtained as sum of elastic and inelastic discontinuities:
\bea
\Im T_{\Upsilon p}(\nu)  = \theta(\nu - \nu_{el}) \,  {\rm Disc}_{\rm el} T_{\Upsilon p}(\nu) +   \theta(\nu - \nu_{inel}) \,  {\rm Disc}_{\rm inel} T_{\Upsilon p}(\nu).
\label{eq:disctot}
\eea
The elastic discontinuity starts from elastic threshold $s = s_{el} = (M_\Upsilon + M)^2 = 108.13$~GeV$^2$, or equivalently $\nu_{el} = M_\Upsilon M = 8.88$~GeV$^2$, whereas the inelastic discontinuity starts at the $B \bar B$ meson production threshold, corresponding with $s_{inel} = (M + 2 M_B)^2 = 132.18$~GeV$^2$, or equivalently $\nu_{inel} = 20.90$~GeV$^2$. 

Analogous to the $J/\psi$ case~\cite{Gryniuk:2016mpk}, we will parametrize the elastic and inelastic discontinuities of the $\Upsilon p$  forward scattering amplitude by the following 3-parameter forms, for $x = {\rm el/inel}$:
\bea
{\rm Disc}_{x} T_{\Upsilon p}(\nu)  &=& 
C_{x} \left( 1 - \frac{\nu_{x}}{\nu} \right)^{b_{x}}  \left( \frac{\nu}{\nu_{x}} \right)^{a_{x}} ,
\label{eq:discx} 
\eea
where the factors $\sim (1 - \nu_x / \nu)^{b_x}$  
determine the behavior around the respective threshold $\nu_x$, and the 
factors  $\sim \nu^{a_x}$ determine the Regge behavior of the amplitude at large $\nu$. 
In the following we will discuss how we can determine the respective parameters 
appearing in the elastic and inelastic discontinuities. 

We use the vector meson dominance (VMD) assumption to relate the $\Upsilon p$ elastic cross section 
$\sigma_{\Upsilon p}^{el}$ to the total $\gamma p \to \Upsilon p$ photo-production cross section~\cite{Barger:1975ng,Redlich:2000cb}, which fixes the 
discontinuity across the elastic cut as: 
\bea
{\rm Disc}_{\rm el} T_{\Upsilon p}(\nu)  &=& 2 \sqrt{s} \, q_{\Upsilon p} \sigma_{\Upsilon p}^{el} 
=  2 \sqrt{s} \, q_{\Upsilon p}  \left( \frac{M_\Upsilon}{e f_\Upsilon} \right)^2 \left( \frac{q_{\gamma p}}{q_{\Upsilon p}} \right)^2 \, 
\sigma (\gamma p \to \Upsilon p), 
\label{eq:sigmael}
\eea
with electric charge $e$ given through $\alpha = e^2 / (4 \pi) \simeq 1/137$, and  where $f_\Upsilon$ is the $\Upsilon$ decay constant, which is obtained from the $\Upsilon \to e^+ e^-$ decay as 
\begin{eqnarray}
\Gamma_{\Upsilon \to ee} = \frac{4 \pi \alpha^2}{3} \frac{f_\Upsilon^2}{M_\Upsilon}.
\end{eqnarray}
The experimental value $\Gamma_{\Upsilon \to ee} =  1.34$~keV yields $f_\Upsilon = 0.238$~GeV. Furthermore, $q_{\gamma p}$ denotes the magnitude of the 
$\gamma$ three-momentum in the c.m. frame of the $\gamma p \to \Upsilon p$ process:
\bea
q_{\gamma p} = \frac{(s - M^2)}{2 \sqrt{s}}.
\eea

The discontinuity across the inelastic cut, ${\rm Disc}_{inel} T_{\Upsilon p}$, 
is related through the optical theorem to the $\Upsilon \, p \to b \bar b X$ inelastic cross section 
$\sigma_{\Upsilon p}^{inel}$, which is again related, using VMD, to the corresponding $\gamma p \to b \bar b X$ photo-production cross section:
\bea
{\rm Disc}_{\rm inel} T_{\Upsilon p}(\nu) = 2 \sqrt{s} \, q_{\Upsilon p} \, \sigma_{\Upsilon p}^{inel}    
 = 2 \sqrt{s} \, q_{\Upsilon p} \,  \left( \frac{M_\Upsilon}{e f_\Upsilon} \right)^2  \left( \frac{q_{\gamma p}}{q_{\Upsilon p}} \right)^2 \,  \sigma (\gamma p \to b \bar b X). 
\label{eq:sigmainel}
\eea

Having fixed the imaginary part of $T_{\Upsilon p}$,  
the real part of $T_{\Upsilon p}$ is related to its imaginary part
through a once-subtracted forward dispersion relation:
\beq
\Re T_{\Upsilon p}(\nu) = T_{\Upsilon p}(0) + \frac{2}{\pi} \nu^2 \int_{\nu_{el}}^\infty d
\nu^\prime \frac{1}{\nu^\prime} \frac{\Im T_{\Upsilon p}(\nu^\prime)}{\nu^{\prime \, 2} - \nu^2},
\label{eq:disp}
\eeq
with $T_{\Upsilon p}(0) $ the subtraction constant at $\nu = 0$. In this work, the subtraction constant will be obtained by performing a fit of the 
differential $\gamma p \to \Upsilon p$  photo-production cross section data at $t=0$, which is related to $T_{\Upsilon p}$ as:
\beq
\frac{d \sigma}{dt} \biggr|_{t = 0} (\gamma p \to \Upsilon p) 
= \left( \frac{e f_\Upsilon}{M_\Upsilon} \right)^2  \frac{1}{64 \, \pi \, s \, q_{\gamma p}^2} \, \big| T_{\Upsilon p}(\nu) \big|^2.
\label{eq:dsigmadt0_gapjpsip}
\eeq

The real part of the forward scattering amplitude at threshold $T_{\Upsilon p}(\nu_{el}) $ is directly related to the $\Upsilon p$ scattering length 
$a_{\Upsilon p}$ as:
\bea
T_{\Upsilon p}(\nu = \nu_{el}) = 8 \pi (M + M_\Upsilon) \, a_{\Upsilon p}. 
\eea
Analogously to the $J/\Psi$ case, 
we may relate a positive $\Upsilon p$ scattering length, corresponding to an attractive interaction, 
to an $\Upsilon$ binding energy $B_\Upsilon$ in nuclear matter, using a linear density approximation~\cite{Kaidalov:1992hd}:
\begin{eqnarray}
B_\Upsilon \simeq \frac{8 \pi (M + M_\Upsilon) a_{\Upsilon p}}{4 M M_\Upsilon} \, \rho_{nm},
\label{eq:nmbe}
\end{eqnarray}
where $\rho_{nm} \simeq 0.17$~fm$^{-3}$ denotes the nuclear matter density.



\section{Results and discussion}

In order to empirically determine the $\Upsilon p$ scattering length $a_{\Upsilon p}$, the strategy we use in this work follows our previous analysis for 
the $J/\psi p$ system~\cite{Gryniuk:2016mpk}. The data on the $\gamma p \to \Upsilon p$ and $\gamma p \to b \bar b X$ total cross sections 
are parameterized according to the three-parameter forms of Eq.~(\ref{eq:discx}) inserted into Eqs.~(\ref{eq:sigmael}) and (\ref{eq:sigmainel}).
This determines the total imaginary part in the dispersion integral of Eq.~(\ref{eq:disp}). The real part is then calculated from the dispersion integral  
and the subtraction constant is extracted from a fit to the differential $\gamma p \to \Upsilon p$ cross section extrapolated to $t = 0$, using Eq.~(\ref{eq:dsigmadt0_gapjpsip}).
 
At present, the exclusive $\Upsilon$ photo-production database
consists of four data points from HERA~\cite{Adloff:2000vm,Breitweg:1998ki,Chekanov:2009zz}
(see Fig.~\ref{fig:sigmatot}, left panel). 
Furthermore at Large Hadron Collider (LHC) energies, the $\gamma p \to \Upsilon p$ cross section has been extracted from central pp production data at LHCb~\cite{Aaij:2015kea} 
and from ultra-peripheral pPb collisions at  CMS~\cite{Sirunyan:2018sav}
For high energies $\sqrt{s} = W\sim100$ GeV, the differential cross section data shows an exponential t-dependence 
%corresponding to the dependence
\beq
\frac{d \sigma}{dt} (\gamma p \to \Upsilon p)
\;=\; A \cdot e^{Bt}, \quad \quad 
A = \frac{d \sigma}{dt} \biggr|_{t = 0} (\gamma p \to \Upsilon p) ,
\label{eq:bdef}
\eeq
with an empirical slope parameter $B(W = 100$~GeV$)=4.5\pm0.5$~GeV$^{-2}$~\cite{Chekanov:2009zz}. 
The exponential dependence of Eq.~(\ref{eq:bdef}) allows us to express the extrapolated value
 of the differential cross section at $t=0$ as
\beq
A  \simeq B e^{- B t_{\rm min}} \cdot\sigma (\gamma p \to \Upsilon p),
\label{eq:brel}
\eeq
where
\beq
t_\mathrm{min} = M_\Upsilon^2 - 2q_{\gamma p} \left(\sqrt{q_{\Upsilon p}^2 + M_\Upsilon^2} - q_{\Upsilon p}\right)
\eeq
is the minimum (modulo) physical momentum transfer, corresponding to the forward scattering ($\theta_{\gamma \Upsilon}=0$).

The inclusive $b \bar b$ photo-production database is represented by one data point from HERA~\cite{Adloff:1999nr}.
Additionally, the lower energy cross section upper limit from EMC~\cite{Aubert:1981gx}
is added to guide the low-energy behaviour (see Fig.~\ref{fig:sigmatot}, right panel).

\begin{figure}
\includegraphics[width=0.49\textwidth]{si_y.pdf}
\includegraphics[width=0.49\textwidth]{si_bbX.pdf}
\caption{W-dependence of the $\gamma p \to \Upsilon p$ (left) and $\gamma p \to b \bar b X$ (right) total cross sections.
The curve is the result of our global fit with parameters given in Table~\ref{tab:fits}.
The elastic $\Upsilon$ photo-production data are from HERA: H1~\cite{Adloff:2000vm}
and ZEUS~\cite{Breitweg:1998ki, Chekanov:2009zz}, and from LHC: LHCb~\cite{Aaij:2015kea} 
and CMS~\cite{Sirunyan:2018sav}. 
The simulated EIC data points shown are based on the differential in $t$ cross section points:
the fitted $Ae^{Bt}$ is integrated between $t_\mathrm{min}$ and $t_\mathrm{max}$.
The error bars of the EIC datapoints are propagated through the resulting covariance matrix of the $A$ and $B$ parameters.
The open beauty production data points are from HERA~\cite{Adloff:1999nr}, EMC~\cite{Aubert:1981gx}.}
\label{fig:sigmatot}
\end{figure}



The present database is insufficient to perform a high-quality fit using the forms of Eq.~(\ref{eq:discx}) with all the parameters unconstrained.
We may reasonably expect however, for a diffractive process, that the $a_x$ values are in general similar to those for the $J/\psi$.
The lack of low-energy data prevents a direct determination of the low-energy slope parameters $b_x$ of the cross sections at present.
We thus start by simply fixing all the low- and high-energy slope parameters to the $J/\psi$ values from~\cite{Gryniuk:2016mpk}:
\bea
a_{\rm el} = 1.38, &\quad& a_{\rm inel} = 1.20, \\
b_{\rm el} = 1.27, &\quad& b_{\rm inel} = 3.53 .
\eea
We then fit the elastic normalization constant $C_{\rm el}$ to the four data points of the elastic $\Upsilon$ photo-production cross section, as shown in Fig.~\ref{fig:sigmatot} (left panel). 
In order to fit the inelastic normalization constant $C_{\rm inel}$, we make the observation from Fig.~\ref{fig:sigmatot} that around $W = 100$~GeV, the HERA data for the inelastic cross section are around 2 orders of magnitude larger than their elastic counterparts. We can thus safely assume that the amplitude $T_{\Upsilon p}$ entering the differential cross section in Eq.~(\ref{eq:dsigmadt0_gapjpsip}) is dominated by the imaginary part of the inelastic process, proportional to $C_{\rm inel}$. 
By then relating the cross section ratio of Eq.~(\ref{eq:brel}) to the slope parameter $B$ at $W=100$ GeV, 
we obtain the relation:
\beq
B (W = 100~{\rm GeV}) \simeq  
 \frac{C_{\rm inel}^2}{C_{\rm el}} \left[ \frac{1}{16 \pi W^2} \left( \frac{\nu}{\nu_{\rm inel}} \right)^{2 a_{\rm inel}} 
\left( \frac{\nu_{\rm el}}{\nu} \right)^{a_{\rm el}} \right]_{W = 100~{\rm GeV} } . 
\label{eq:Crel}
\eeq
Using the empirical slope parameter at $W = 100$~GeV and the fit value of $C_{\rm el}$, allows then to fix the normalization constant $C_{\rm inel}$. 
The final values of the total cross section parameters are listed in Table~\ref{tab:fits}. 
We obtain for the elastic photo-production total cross section data $\chi^2/{\rm d.o.f.} = 0.3$ (11 points, 2 parameters).

\begin{table}[h]
\begin{tabular*}{\textwidth}{c @{\extracolsep{\fill}} cccc}
\hline
\hline
& \quad $x$ = el \quad & \quad $x$ = inel \quad\\
\hline
$C_x$ & $(14.81\pm2.10)\times 10^{-3}$ & $18.73\pm1.33$ \\
$b_x$ & $1.27$ & $3.53$ \\
$a_x$ & $1.37$ & $1.2$ \\
\hline
\hline
\end{tabular*}
\caption{Fit results for the coefficients entering the elastic discontinuity (second column, $x = el$), 
and the inelastic discontinuity (third column, $x = inel$).
The $b_x$ parameters are fixed from $J/\psi$ case result~\cite{Gryniuk:2016mpk}.
%\textcolor{red}{The errors are based on the covariance matrix.}
% The given uncertainties for the $a_{\rm el}$ and $C_{\rm el}$ parameters
% are the {\it uncorrelated} linear estimates based on the four data points of the elastic photoproduction.
% The uncertainty of $a_{\rm inel}$ is obtained similarly based on the two inelastic photoproduction points.
The $C_{\rm inel}$ parameter uncertainty is simply estimated as half of the relative uncertainty of $C_{\rm el}$
based on relation of Eq.~(\ref{eq:Crel}).
}
\label{tab:fits}
\end{table}


Having fixed the imaginary part of the amplitude $T_{\Upsilon p}$, we evaluate the dispersion integral of Eq.~(\ref{eq:disp}). 
The real part is then calculated from the dispersion relation up to the subtraction constant. 
Whereas at high energies (i.e. HERA energies) the differential cross section is largely dominated by its imaginary part, we get sizeable sensitivity to the real part at low to intermediate energies. We will explore in the following how to extract the subtraction constant from a fit to the 
 differential $\gamma p \to \Upsilon p$ cross section data at an Electron-Ion Collider (EIC) at two energy settings, corresponding to the range $W < 30$~GeV and $W < 50$~GeV.    

In order to perform a feasibility study of such forthcoming data, we will consider three scenarios for the subtraction constant in order to 
explore the data sensitivity to its extraction. 
The simplest scenario corresponds with having zero value of the subtraction constant. The real part of the $\Upsilon p$ scattering amplitude is then fully determined by its imaginary part through the dispersion integral. The resulting value of the $\Upsilon p$ scattering length is then extremely small, around $a_{\Upsilon p} \simeq 0.6 \times 10^{-3}$~fm. 

A second scenario is to estimate the subtraction constant by a scaling from its value for the $J/\psi p$ scattering, which was obtained from a fit to data in~\cite{Gryniuk:2016mpk} as $T_{J/\psi p }(0) \simeq 22.5 \pm 2.5$. 
By observing that at high energies the normalizations of both the $J/\psi p$ and $\Upsilon p$ scattering amplitudes are completely driven by their inelastic discontinuities, and by making the strong assumption that the subtraction constants scale in the same way, %we obtain the estimate:
we thus estimate them to be roughly the same:
\beq
T_{\Upsilon p}(0) = T_{J/\psi p }(0) \cdot C_{\rm inel}^\Upsilon / C_{\rm inel}^{J/\psi} \approx T_{J/\psi p }(0).
\eeq
% which corresponds with a scattering length $a_{\Upsilon p} \simeq 0.016$~fm. 

We also consider a third, theoretically more motivated, scenario, in which an estimate of the threshold scattering amplitude 
is obtained by considering, in the leading approximation,  the heavy bottomonium 
as a Coulombic bound state~\cite{Peskin:1979va, Kaidalov:1992hd}:
\beq
T_{\Upsilon p}(\nu_{el}) = \frac{64\pi^3}{3^6}7 M_{\Upsilon} M_p^2 a_0^3 ,
\eeq
where we use the Bohr radius $a_0$ of the $\Upsilon$
\beq
a_0^{-1} \approx \frac{4}{3}\alpha_s \frac{m_b}{2} \approx (0.85)^{-1} \,\mathrm{GeV},
\eeq
with the values of the strong coupling $\alpha_s\approx 0.37$ and the bottom quark mass $m_b\approx 4.76$ GeV at the corresponding scale.
This estimate yields $T_{\Upsilon p}(\nu_{el}) \simeq 98$ or equivalently for the s-wave scattering length $a_{\Upsilon p} \simeq 0.07$~fm. Using the dispersion integral of Eq.~(\ref{eq:disp}) to relate the amplitudes at $\nu = \nu_{\rm el}$ and $\nu = 0$, we then obtain 
$T_{\Upsilon p}(0) \simeq 97$. 

In Table~\ref{tab:scattlength}, we show the values for the $s$-wave scattering length $a_{\Upsilon p}$ and nuclear matter binding energy $B_\Upsilon$ corresponding to values of $T_{\Upsilon p}(0)$ in the three scenarios discussed above. The uncertainties correspond to the simulated EIC 
$\gamma p \to \Upsilon p$ data for two beam settings as discussed below. 
In Fig.~\ref{fig:psip_psip}, we show the $W$-dependence, in our dispersive formalism, of the real and imaginary parts of the $\Upsilon p$ scattering amplitude $T_{\Upsilon p}$, for the three choices of the subtraction constant discussed above.
  

\begin{table}[h]
\begin{tabular*}{\textwidth}{c @{\extracolsep{\fill}} cccc}
\hline
\hline
\quad beam setting \quad & \quad $T_{\Upsilon p}(0)$ \quad &  \quad $T_{\Upsilon p}(\nu = \nu_{el})$
 \quad & \quad $a_{\Upsilon p}$ (in fm) \quad  & \quad $B_{\Upsilon}$ (in MeV) \quad \\ 
\hline
1 &$0 \pm 0.4$ & $0.76 \pm 0.42$ & $(0.57 \pm 0.32)\times 10^{-3}$ & $0.028 \pm 0.016$ \\
&$21 \pm 1.2$ & $21.76 \pm 1.15$ & $(16.4 \pm 0.9)\times 10^{-3}$ & $0.80 \pm 0.04$ \\
&$97 \pm 2.7$ & $97.76 \pm 2.66$ & $(73.8 \pm 2.0)\times 10^{-3}$ & $3.60 \pm 0.10$ \\
\hline
2 &$0 \pm 1$ & $\simeq 0$ & $\simeq 0$ \\
&$22 \pm 3$ & $0.018 \pm 0.002$ & $0.9 \pm 0.1$ \\
&$97 \pm 8$ & $0.074 \pm 0.006$ & $4.0 \pm 0.3$ \\
% \quad  $0$ \quad & $1.30$ & \quad  $0.003$ \quad & \quad  $0.2$ \quad  \\
% \quad  $22.45 \pm 2.45$ \quad & $23.74 \pm 2.59$ & \quad $0.046 \pm 0.005$ \quad & \quad  $2.7 \pm 0.3$ \quad \\ 
% \quad  $45$ \quad & $46.30$ & \quad $0.090$ \quad & \quad  $5.2$ \quad  \\ 
\hline
\hline
\end{tabular*}
\caption{Values of 
the subtraction term $T_{\Upsilon p}(0)$ (first column), 
the corresponding values of the threshold amplitude $T_{\Upsilon p}(\nu_{\rm el})$ (second column), 
the corresponding $\Upsilon p$ s-wave scattering lengths $a_{\Upsilon p}$ (third column), 
and the corresponding $\Upsilon$-nuclear matter binding energy $B_\Upsilon$, according to Eq.~(\ref{eq:nmbe}) (fourth column).
The uncertainty estimates are propagated based on the generated EIC differential cross section data points.
}
\label{tab:scattlength}
\end{table}

\begin{figure}[h]
\includegraphics[width=0.49\textwidth]{t_y.pdf}
\includegraphics[width=0.49\textwidth]{reimt_y.pdf}
\caption{Left panel:
Imaginary part (dotted curve) and real part of the forward scattering amplitude $T_{\Upsilon p}$ as function of W.
The real part is shown for different values of the subtraction constant as indicated on the figure.
Right panel: corresponding ratios of real over imaginary parts plotted starting from inelastic threshold.}
\label{fig:psip_psip}
\end{figure}

We used the Argonne l/A-event Generator~\cite{git:lager} to simulate a realistic event sample for the $\gamma p \to \Upsilon p$ process at the EIC.
Please refer to Appendix~\ref{apx-evgen} for more details on the exact implementation of the simulation.
We evaluated three scenarios for the value of the the s-wave scattering length $a_{\Upsilon p}$,
considering both a medium-energy and high-energy EIC configuration.
The medium-energy configuration (setting 1) has a 10~GeV electron beam incident on a 100~GeV proton beam 
($\sqrt{s} = 63$~GeV), while the high-energy configuration (setting 2) has a 18~GeV electron beam incident 
on a 275~GeV proton beam ($\sqrt{s} = 140$~GeV)\footnote{These settings correspond to nominal configurations for the EIC design.}.
We assumed a total integrated luminosity of 100~fb$^{-1}$ for each of the settings, which corresponds to 116 
days at 10$^{34}$cm$^{-2}$s$^{-1}$. {\color{red} TODO: DOUBLE-CHECK THIS NUMBER.} 
We simulated both the $\Upsilon \to e^+e^-$ and $\Upsilon \to \mu^+\mu^-$ decay channels, and only considered events where we fully detect the exclusive final state.
We assumed nominal EIC detector parameters in line with the EIC 
white paper, where we have lepton detection for pseudo-rapidities 
between $-5 < \eta_l < 5$, and recoil proton detection for angles 
$\theta_p > 2$~mrad. 
Furthermore, we assumed we can reconstruct the scattered electron for 
$y = P.q/P.k$ between $0.01 < y < 0.8$, and we ensured a 
quasi-real regime by requiring that $Q^2 < 1$~GeV$^{2}$.

%For the differential cross section we assume an exponential $t$-dependence according to Eq.~(\ref{eq:bdef}) and normalize its 
%value at $t = 0$, given by $A$, to our theoretical estimate for the three choices of the subtraction constant 
%$T_{\Upsilon p} (0)$ discussed above. We adjust the $t$-slope $B$ such that it returns the theoretical $\gamma p \to \Upsilon p$ 
%total cross section upon integration.  
%{\color{red} (CHECK THAT THIS IS WHAT IS DONE IN THE SIMULATION, especially when $-t_{min} \neq 0$)}

In Fig.~\ref{fig:dsigmadt}, we show the simulated results for the $t$-dependence of the $\gamma p \to \Upsilon p$ 
differential cross sections for different values of $W$,  
corresponding to the two EIC beam settings.


\begin{figure}[h]
\includegraphics[width=0.49\textwidth]{{dsdt_y_eic1_W_11.5}.pdf}
\includegraphics[width=0.49\textwidth]{{dsdt_y_eic2_W_11.5}.pdf}
\includegraphics[width=0.49\textwidth]{{dsdt_y_eic1_W_21.5}.pdf}
\includegraphics[width=0.49\textwidth]{{dsdt_y_eic2_W_21.5}.pdf}
\includegraphics[width=0.49\textwidth]{{dsdt_y_eic1_W_35.5}.pdf}
\includegraphics[width=0.49\textwidth]{{dsdt_y_eic2_W_35.5}.pdf}
\caption{$t$-dependence of the $\gamma p \to \Upsilon p$ differential cross section 
for different values of the subtraction constant $T_0 \equiv T_{\Upsilon p} (0)$ as indicated on the figure. The data points are simulated based on the theoretical elastic $\Upsilon$ photo-production cross section, assuming an exponential $t$-dependence. The bands represent the uncertainty propagated based on the EIC simulated data sample, assuming one-parameter fit of $T_{\Upsilon p}(0)$.}
\label{fig:dsigmadt}
\end{figure}


\begin{figure}
%\includegraphics[width=0.49\textwidth]{dsdt_y.pdf}
%\includegraphics[width=0.49\textwidth]{dsdt_y_close.pdf}
\includegraphics[width=0.49\textwidth]{dsdt_y_eic1.pdf}
\includegraphics[width=0.49\textwidth]{dsdt_y_close_eic1.pdf}
\includegraphics[width=0.49\textwidth]{dsdt_y_eic2.pdf}
\includegraphics[width=0.49\textwidth]{dsdt_y_close_eic2.pdf}
\caption{W-dependence of the $\gamma p \to \Upsilon p$ differential cross section, extrapolated to the forward direction ($t=0$), 
for different values of the subtraction constant $T_0 \equiv T_{\Upsilon p} (0)$ in the forward $\Upsilon p$ scattering amplitude. The (black circles) data points are obtained from the elastic $\Upsilon$ photo-production cross section from HERA by using the empirically measured slope parameter, using Eq.~(\ref{eq:brel}). The bands represent the uncertainty propagated based on the EIC generated data points, assuming one-parameter fit of $T_{\Upsilon p}(0)$.}
\label{fig:dsigmadt0}
\end{figure}


\begin{figure}
% \includegraphics[width=0.49\textwidth]{b_slope.pdf}
\includegraphics[width=0.49\textwidth]{b_slope_eic1.pdf}
\includegraphics[width=0.49\textwidth]{b_slope_eic2.pdf}
\includegraphics[width=0.49\textwidth]{b_slope_jpsi.pdf}
\caption{
% W-dependence of the left hand side of relation of Eq.~\ref{eq:Crel},
The W-dependence of the B slope parameter introduced in relation (\ref{eq:bdef}),
for different values of the subtraction constant $T_{\Upsilon p} (0)$ in the forward $\Upsilon p$ (top) and the forward J/$\psi p $ (bottom) scattering amplitudes.
The value at $W=100$ GeV for the $\Upsilon$ case is $B=4.5\pm0.5$~GeV$^{-2}$~\cite{Chekanov:2009zz}.
The curves shown are obtained by solving Eq.~(\ref{eq:brel}) for $B$.
The $B$ slope data points shown for the J/$\psi$ production are from HERA~\cite{Chekanov:2002xi}.
}
\label{fig:bslope}
\end{figure}

\section{Conclusion}

\section*{Acknowledgements}
The work of OG and MV was supported by the Deutsche Forschungsgemeinschaft (DFG, German Research Foundation),
in part through the Collaborative Research Center [The Low-Energy Frontier of the Standard
Model, Projektnummer 204404729 - SFB 1044], and in part through the Cluster of Excellence
[Precision Physics, Fundamental Interactions, and Structure of Matter] (PRISMA$^+$ EXC
2118/1) within the German Excellence Strategy (Project ID 39083149).
This work is supported in part by the US DOE contract DE-AC02-06CH11357.


\appendix
\section{Event Generation\label{apx-evgen}}
In order to simulate a realistic event sample for $\Upsilon$ events at the EIC,
we added the formalism of this paper to the Argonne l/A-event Generator (\textsc{lager}) \cite{git:lager}
\textsc{Lager} is a modular accept-reject generator capable of simulating both fixed-target and collider kinematics. Below we describe the model components used to obtain the event samples for this work.

\subsection{Differential electro-production cross section}
The differential cross section for the process ($e p \to e^\prime \gamma^* p \to e^\prime \Upsilon p$) can be written as,
\beq
\frac{d\sigma}{dQ^2dydt}(e p \to e^\prime \Upsilon p) = 
\Gamma_T(1+\epsilon R)
\frac{d\sigma}{dt}(\gamma^* p \to \Upsilon p),
\eeq
with transverse virtual photon flux $\Gamma_T$, virtual photon polarization $\epsilon$, and 
$R\equiv\sigma_L/\sigma_T$ parameterized as in Ref.~\cite{Martynov:2002ez},
\beq
R(Q^2) = \left(\frac{A M_\Upsilon^2 + Q^2}{A M_\Upsilon^2}\right)^{n_1} - 1.
\eeq
We use the values for parameters ($A$, $n_1$) as determined for $J/\psi$ production in Ref.~\cite{Fiore:2009xk}.
In order to estimate the unknown $Q^2$ dependence of the differential $\gamma^* p \to \Upsilon p$ cross section, we use the following factorized ansatz,
\beq
\frac{d\sigma}{dt}(\gamma^* p \to \Upsilon p) = D(Q^2) \frac{d\sigma}{dt}(\gamma p \to \Upsilon p),
\eeq
where for $D$ we assumed a dipole-like form-factor, similar to what is typically
done in a vector meson dominance model (VMD),
\beq
D(Q^2) = \left(\frac{M_\Upsilon^2}{M_\Upsilon^2 + Q^2}\right)^{n_2}.
\eeq
This formula deviates from its standard VMD form through the value for $n_2$, which was
tuned to optimally describe the $Q^2$ dependence for exclusive $\rho$ production in a wide range
of kinematic regions~\cite{Airapetian:2000ni,Adams:1997bh,Tytgat:2011us,Liebing:2004us}.
Note that this assumption has very little impact on the projections in this work,
as we only consider quasi-real events.

\subsection{Differential cross section for $\Upsilon$ production}
In order to determine the slope $B$ of the $t$-dependence of the differential cross section for 
the $\gamma p \to \Upsilon p$ process, we numerically solve the transcendental equation \eqref{eq:bdef},
\beq
\frac{d\sigma}{dt}(\gamma p \to \Upsilon p) = A \cdot e^{Bt}.
\eeq
Note that both  normalization $A$ and slope $B$ depend on the choice of the subtraction
constant $T_{\Upsilon p} (0)$, while the total integrated cross section 
$\sigma(\gamma p \to \Upsilon p)$ is independent of the subtraction constant.

\subsection{Angular dependence of the decay leptons}
We included both the $\Upsilon \to e^+e^-$ and $\Upsilon \to \mu^+\mu^-$ decay channels
in our simulation, using $s$-channel helicity conservation (SCHC) to describe the
angular distribution for a vector meson decaying into two fermions~\cite{Breitweg:1998nh,Chekanov:2002xi,Schilling:1973ag}.
\beq
\mathcal{W}(\cos\theta_\text{CM}) = 
\frac{3}{8}(1+r^{04}_{00}+(1-3r_{00}^{04})\cos^2\theta_\text{CM}),
\eeq
where we relate the spin-density matrix element $r^{04}_{00}$ to $R$ as,
\beq
R = \frac{1}{\epsilon}\frac{r^{04}_{00}}{1-r^{04}_{00}}.
\eeq

\begin{thebibliography}{99}

\bibitem{Kharzeev:1995ij} 
  D.~Kharzeev,
  %``Quarkonium interactions in QCD,''
  Proc.\ Int.\ Sch.\ Phys.\ Fermi {\bf 130}, 105 (1996).
%  doi:10.3254/978-1-61499-215-8-105
 % [nucl-th/9601029].
  %%CITATION = doi:10.3254/978-1-61499-215-8-105;%%

\bibitem{Voloshin:2007dx} 
  M.~B.~Voloshin,
  %``Charmonium,''
  Prog.\ Part.\ Nucl.\ Phys.\  {\bf 61}, 455 (2008).
 % doi:10.1016/j.ppnp.2008.02.001
 % [arXiv:0711.4556 [hep-ph]].
  %%CITATION = doi:10.1016/j.ppnp.2008.02.001;%%
  
\bibitem{Hosaka:2016ypm} 
  A.~Hosaka, T.~Hyodo, K.~Sudoh, Y.~Yamaguchi and S.~Yasui,
  %``Heavy Hadrons in Nuclear Matter,''
  arXiv:1606.08685 [hep-ph].
  %%CITATION = ARXIV:1606.08685;%%

\bibitem{Brodsky:1989jd} 
  S.~J.~Brodsky, I.~A.~Schmidt and G.~F.~de Teramond,
  %``Nuclear Bound Quarkonium,''
  Phys.\ Rev.\ Lett.\  {\bf 64}, 1011 (1990).
%  doi:10.1103/PhysRevLett.64.1011
  %%CITATION = doi:10.1103/PhysRevLett.64.1011;%%
  
\bibitem{Wasson:1991fb} 
  D.~A.~Wasson,
  %``Comment on `Nuclear bound quarkonium.',''
  Phys.\ Rev.\ Lett.\  {\bf 67}, 2237 (1991).
%  doi:10.1103/PhysRevLett.67.2237
  %%CITATION = doi:10.1103/PhysRevLett.67.2237;%%  
  
    
  
\bibitem{Luke:1992tm} 
  M.~E.~Luke, A.~V.~Manohar and M.~J.~Savage,
  %``A QCD Calculation of the interaction of quarkonium with nuclei,''
  Phys.\ Lett.\ B {\bf 288}, 355 (1992). 
%  doi:10.1016/0370-2693(92)91114-O
%  [hep-ph/9204219].
  %%CITATION = doi:10.1016/0370-2693(92)91114-O;%%  

  
\bibitem{Peskin:1979va} 
  M.~E.~Peskin,
  %``Short Distance Analysis for Heavy Quark Systems. 1. Diagrammatics,''
  Nucl.\ Phys.\ B {\bf 156}, 365 (1979).
 % doi:10.1016/0550-3213(79)90199-8
  %%CITATION = doi:10.1016/0550-3213(79)90199-8;%%

\bibitem{Sugiura:2017vks}
T.~Sugiura, Y.~Ikeda and N.~Ishii,
%``Charmonium-nucleon interactions from the time-dependent HAL QCD method,''
EPJ Web Conf. \textbf{175}, 05011 (2018).
%doi:10.1051/epjconf/201817505011
%[arXiv:1711.11219 [hep-lat]].


\bibitem{Polyakov:2018aey}
M.~V.~Polyakov and P.~Schweitzer,
%``Determination of $J/\psi$ chromoelectric polarizability from lattice data,''
Phys. Rev. D \textbf{98}, no.3, 034030 (2018).
%doi:10.1103/PhysRevD.98.034030
%[arXiv:1801.08984 [hep-ph]].


\bibitem{Aaij:2015tga}
R.~Aaij \textit{et al.} [LHCb],
%``Observation of $J/\psi p$ Resonances Consistent with Pentaquark States in $\Lambda_b^0 \to J/\psi K^- p$ Decays,''
Phys. Rev. Lett. \textbf{115}, 072001 (2015). 
%doi:10.1103/PhysRevLett.115.072001
%[arXiv:1507.03414 [hep-ex]].

\bibitem{Eides:2015dtr}
M.~I.~Eides, V.~Y.~Petrov and M.~V.~Polyakov,
%``Narrow Nucleon-$\psi(2S)$ Bound State and LHCb Pentaquarks,''
Phys. Rev. D \textbf{93}, no.5, 054039 (2016)
%doi:10.1103/PhysRevD.93.054039
%[arXiv:1512.00426 [hep-ph]].

\bibitem{Behnke:2015qja}
O.~Behnke, A.~Geiser and M.~Lisovyi,
%``Charm, Beauty and Top at HERA,''
Prog. Part. Nucl. Phys. \textbf{84}, 1-72 (2015).
%doi:10.1016/j.ppnp.2015.06.002
%[arXiv:1506.07519 [hep-ex]].



\bibitem{Gryniuk:2016mpk} 
  O.~Gryniuk and M.~Vanderhaeghen,
  %``Accessing the real part of the forward $J/\psi$-p scattering amplitude from $J/\psi$ photoproduction on protons around threshold,''
  Phys.\ Rev.\ D {\bf 94}, no. 7, 074001 (2016)
%  doi:10.1103/PhysRevD.94.074001
%  [arXiv:1608.08205 [hep-ph]].
  %%CITATION = doi:10.1103/PhysRevD.94.074001;%%

\bibitem{Barger:1975ng} 
  V.~D.~Barger and R.~J.~N.~Phillips,
  %``Properties of psi n Scattering,''
  Phys.\ Lett.\ B {\bf 58}, 433 (1975).
%  doi:10.1016/0370-2693(75)90582-1
  %%CITATION = doi:10.1016/0370-2693(75)90582-1;%%
  

\bibitem{Redlich:2000cb} 
  K.~Redlich, H.~Satz and G.~M.~Zinovjev,
  %``Photoproduction constraints on J / psi nucleon interactions,''
  Eur.\ Phys.\ J.\ C {\bf 17}, 461 (2000)
%  doi:10.1007/s100520000488
 % [hep-ph/0003079].
  %%CITATION = doi:10.1007/s100520000488;%%

\bibitem{Kaidalov:1992hd} 
  A.~B.~Kaidalov and P.~E.~Volkovitsky,
  %``Heavy quarkonia interactions with nucleons and nuclei,''
  Phys.\ Rev.\ Lett.\  {\bf 69}, 3155 (1992).
  % doi:10.1103/PhysRevLett.69.3155
  %%CITATION = doi:10.1103/PhysRevLett.69.3155;%%
 
 
%\cite{Adloff:2000vm}
\bibitem{Adloff:2000vm} 
  C.~Adloff {\it et al.} [H1 Collaboration],
  %``Elastic photoproduction of J / psi and Upsilon mesons at HERA,''
  Phys.\ Lett.\ B {\bf 483}, 23 (2000)
  % doi:10.1016/S0370-2693(00)00530-X
%  [hep-ex/0003020].
  %%CITATION = doi:10.1016/S0370-2693(00)00530-X;%%


%\cite{Breitweg:1998ki}
\bibitem{Breitweg:1998ki} 
  J.~Breitweg {\it et al.} [ZEUS Collaboration],
  %``Measurement of elastic Upsilon photoproduction at HERA,''
  Phys.\ Lett.\ B {\bf 437}, 432 (1998)
  % doi:10.1016/S0370-2693(98)01081-8
%  [hep-ex/9807020].
  %%CITATION = doi:10.1016/S0370-2693(98)01081-8;%%
 

%\cite{Chekanov:2009zz}
\bibitem{Chekanov:2009zz} 
  S.~Chekanov {\it et al.} [ZEUS Collaboration],
  %``Exclusive photoproduction of upsilon mesons at HERA,''
  Phys.\ Lett.\ B {\bf 680}, 4 (2009)
  % doi:10.1016/j.physletb.2009.07.066
%  [arXiv:0903.4205 [hep-ex]].
  %%CITATION = doi:10.1016/j.physletb.2009.07.066;%%
 

%\cite{Adloff:1999nr}
\bibitem{Adloff:1999nr} 
  C.~Adloff {\it et al.} [H1 Collaboration],
  %``Measurement of open beauty production at HERA,''
  Phys.\ Lett.\ B {\bf 467}, 156 (1999)
  Erratum: [Phys.\ Lett.\ B {\bf 518}, 331 (2001)]
  % doi:10.1016/S0370-2693(99)01099-0, 10.1016/S0370-2693(01)01035-8
%  [hep-ex/9909029].
  %%CITATION = doi:10.1016/S0370-2693(99)01099-0, 10.1016/S0370-2693(01)01035-8;%%
 

%\cite{Aubert:1981gx}
\bibitem{Aubert:1981gx} 
  J.~J.~Aubert {\it et al.} [European Muon Collaboration],
  %``Observation of Wrong Sign Trimuon Events in 250-{GeV} Muon - Nucleon Interactions,''
  Phys.\ Lett.\  {\bf 106B}, 419 (1981).
  % doi:10.1016/0370-2693(81)90655-9
  %%CITATION = doi:10.1016/0370-2693(81)90655-9;%%
 
\bibitem{Peskin:1979va} 
  M.~E.~Peskin,
  %``Short Distance Analysis for Heavy Quark Systems. 1. Diagrammatics,''
  Nucl.\ Phys.\ B {\bf 156}, 365 (1979).
  % doi:10.1016/0550-3213(79)90199-8
  %%CITATION = doi:10.1016/0550-3213(79)90199-8;%%


\bibitem{Chekanov:2002xi} 
  S.~Chekanov {\it et al.} [ZEUS Collaboration],
  %``Exclusive photoproduction of J / psi mesons at HERA,''
  Eur.\ Phys.\ J.\ C \textbf{24}, 345 (2002)
 % doi:10.1007/s10052-002-0953-7
 % [hep-ex/0201043].
  %%CITATION = doi:10.1007/s10052-002-0953-7;%%


\bibitem{Aaij:2015kea}
R.~Aaij {\it et al.} [LHCb],
%``Measurement of the exclusive Υ production cross-section in pp collisions at $ \sqrt{s}=7 $ TeV and 8 TeV,''
JHEP \textbf{09}, 084 (2015). 
%doi:10.1007/JHEP09(2015)084
%[arXiv:1505.08139 [hep-ex]].


\bibitem{Sirunyan:2018sav}
A.~M.~Sirunyan {\it et al.} [CMS],
%``Measurement of exclusive $\Upsilon$ photoproduction from protons in pPb collisions at $\sqrt{s_\mathrm{NN}} =$ 5.02 TeV,''
Eur. Phys. J. C \textbf{79}, no.3, 277 (2019). 
%doi:10.1140/epjc/s10052-019-6774-8
%[arXiv:1809.11080 [hep-ex]].

\bibitem{git:lager}
S. Joosten, Argonne l/A-event Generator (2020), GitLab Repository,
\url{https://eicweb.phy.anl.gov/monte_carlo/lager}

%\cite{Martynov:2002ez}
\bibitem{Martynov:2002ez}
E.~Martynov {\it et al.},
%``Photoproduction of vector mesons in the soft dipole pomeron model,''
Phys. Rev. D \textbf{67}, 074023 (2003)
doi:10.1103/PhysRevD.67.074023
[arXiv:hep-ph/0207272 [hep-ph]].
%25 citations counted in INSPIRE as of 21 Apr 2020

%\cite{Fiore:2009xk}
\bibitem{Fiore:2009xk}
R.~Fiore {\it et al.},
%``Exclusive J/Psi electroproduction in a dual model,''
Phys. Rev. D \textbf{80}, 116001 (2009)
doi:10.1103/PhysRevD.80.116001
[arXiv:0911.2094 [hep-ph]].
%7 citations counted in INSPIRE as of 21 Apr 2020

%\cite{Airapetian:2000ni}
\bibitem{Airapetian:2000ni}
A.~Airapetian {\it et al.} [HERMES],
%``Exclusive leptoproduction of rho0 mesons from hydrogen at intermediate virtual photon energies,''
Eur. Phys. J. C \textbf{17}, 389-398 (2000)
doi:10.1007/s100520000483
[arXiv:hep-ex/0004023 [hep-ex]].
%99 citations counted in INSPIRE as of 21 Apr 2020
%\cite{Adams:1997bh}
\bibitem{Adams:1997bh}
M.~Adams {\it et al.} [E665],
%``Diffractive production of $\rho^0(770)$ mesons in muon proton interactions at 470-GeV,''
Z. Phys. C \textbf{74}, 237-261 (1997)
doi:10.1007/s002880050386
%113 citations counted in INSPIRE as of 21 Apr 2020
%\cite{Tytgat:2011us}
\bibitem{Tytgat:2011us}
M.~Tytgat,
%``Diffractive Production of rho0 and omega Vector Mesons at HERMES,''
Ph.D. Thesis (2011)
%\cite{Liebing:2004us}
\bibitem{Liebing:2004us}
P.~Liebing,
%``Can the gluon polarization in the nucleon be extracted from HERMES data on single high-p(T) hadrons?,''
doi:10.3204/DESY-THESIS-2004-036 (2004)
%10 citations counted in INSPIRE as of 21 Apr 2020

%\cite{Breitweg:1998nh}
\bibitem{Breitweg:1998nh}
J.~Breitweg {\it et al.} [ZEUS],
%``Exclusive electroproduction of $\rho^0$ and $J/\psi$ mesons at HERA,''
Eur. Phys. J. C \textbf{6}, 603-627 (1999)
doi:10.1007/s100529901051
[arXiv:hep-ex/9808020 [hep-ex]].
%258 citations counted in INSPIRE as of 21 Apr 2020
%\cite{Chekanov:2002xi}
%\bibitem{Chekanov:2002xi}
%S.~Chekanov {\it et al.} [ZEUS],
%``Exclusive photoproduction of $J/\psi$ mesons at HERA,''
%Eur. Phys. J. C \textbf{24}, 345-360 (2002)
%doi:10.1007/s10052-002-0953-7
%[arXiv:hep-ex/0201043 [hep-ex]].
%305 citations counted in INSPIRE as of 21 Apr 2020
%\cite{Schilling:1973ag}
\bibitem{Schilling:1973ag}
K.~Schilling and G.~Wolf,
%``How to analyze vector meson production in inelastic lepton scattering,''
Nucl. Phys. B \textbf{61}, 381-413 (1973)
doi:10.1016/0550-3213(73)90371-4
%242 citations counted in INSPIRE as of 21 Apr 2020

\end{thebibliography}

\end{document}